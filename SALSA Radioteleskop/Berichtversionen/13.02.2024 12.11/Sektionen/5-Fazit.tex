\documentclass[../main.tex]{subfiles}
\begin{document}

    Im Rahmen der Signalanalyse stellen wir ein zeitlich-auflösungsbezogenes Optimum bei einer Integrationszeit von $\SI{60}{\s}$ durch optische Betrachtung fest. Der Wechsel vom Modus \enquote{signal} nach \enquote{switched} ist dabei zu empfehlen, da die Peakauflösung klarer ersichtlich wird. \\ 

    Die Annahme der konstanten Bahngeschwindigkeiten konnten wir mit etwas Schwankung bei $\langle V\rangle = \SI{213.33\pm1.606}{\km\per\s}$ feststellen, was zwar auf den ersten Blick von den $V_\odot = \SI{220}{\km\per\s}$ der Sonne abweicht, jedoch aufgrund ausbleibender weiterführender Unsicherheitsanalyse möglicherweise doch mit $V_\odot$ vereinbar ist. \\

    Im letzten Teil des Versuchs haben wir Teile und Strukturen der Milchstraße im Bereich $l\in\{\SI{30}{\degree},\ldots,\SI{215}{\degree}\}$ durch unsere Messungen graphisch rekonstruieren und mithilfe externer Abbildungen validieren können. Da wir jedoch ebenfalls stärker abweichende oder bisher nicht zugeordnete Werte vorliegen haben, empfehlen wir eine erneute Messreihe mit mehrfacher Iteration über einen Raumwinkel zur besseren Mittelung, sowie konsequente Unsicherheitsbetrachtungen hinsichtlich der verwendeten Berechnungsmethoden.   

\end{document}