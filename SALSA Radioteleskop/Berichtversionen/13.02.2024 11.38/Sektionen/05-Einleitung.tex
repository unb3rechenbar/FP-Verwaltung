\documentclass[../main.tex]{subfiles}
\begin{document}

Eines der Hauptziele der Astrophysik ist das Verständnis und das Kartographieren von extraterrestrischen Strukturen wie Planeten, Sternen, Galaxien etc. Da eine direkte Beobachtung mithilfe von Weltraumreisen aktuell keine realistische Option ist, müssen wir uns hierfür auf Teleskope auf der Erdoberfläche oder im Orbit der Erde verlassen. Für unsere eigene Galaxie, die wir nur von innen heraus beobachten können, kann die Kartographie nur indirekt erfolgen.\\

Im Rahmen dieses Experiments demonstrieren wir eine dieser indirekten Möglichkeiten. In der Milchstraße kommt hauptsächlich Wasserstoff vor, welches Hyperfeinübergänge im Frequenzbereich von ca. \SI{1420}{\mega\hertz} besitzt. Die emittierten Wellen haben wegen des Dopplereffekts allerdings eine leicht verschobene Frequenz, was Aufschluss über die Relativgeschwindigkeit der Wasserstoffatome zur Erde gibt.

Unter Annahme von Kreisbahnen der Gaswolken, also dem Wirken einer Zentripetalkraft, kann so der Zusammenhang $V(R)$ zwischen Bahngeschwindigkeit $V$ und Abstand vom Galaxiezentrum $R$ im ersten quadratischen Quadranten bestimmt werden. Nimmt man weiter eine konstante Bahngeschwindigkeit an, lässt sich so sogar den Zusammenhang $R(l)$ zwischen Abstand zum Galaxiezentrum $R$ und galaktischen Längengrad $l$ der Wasserstoffwolken bestimmen: so lässt sich eine Karte der Milchstraße erstellen.\\

\noindent Die Hauptresultate dieses Experiments ist also das Überprüfen der Annahme, dass $V(R)=const.$ und anschließend das Messen des Zusammehangs $R(l)$ für die Längengrade $\SI{0}{\degree}\le l\le \SI{215}{\degree}$. Ein Vergleich mit der aus der bekannten Struktur der Milchstraße liefert weiter ein Indikator für die Genauigkeit der genutzten Methode.

\end{document}