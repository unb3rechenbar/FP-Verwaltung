\documentclass[../main.tex]{subfiles}

\begin{document}

% Methodik: Laserjustage (Grob und fein)

\subsection{Abhängigkeit der Ausgangsleistung von der Eingangsleistung}

% Graphen: out-power-over-in-current.csv plotten [eigentlich zuvor Stromstärke I_in in Leistung umrechnen mit P = R * I^2, aber R = ????]
% Vergleich mit Theorie: zu erwarten kein Lasing bis Schwellleistung P_th, danach in erster Näherung linearer Verlauf P_Laser(P_in) = alpha * (P_in - P_th) [zumindest bei 4-Level-Systemen -> Nd:YAG-Laser]

\subsection{Abhängigkeit der Ausgangsleistung von der Röhrenposition}

% Graphen: power-over-deplacement.csv plotten, fluoreseznezpower von output powerabziehen
% Vergleich mit Theorie: ??

\subsection{Stabilität des Resonators}

% Graphen: outputpower-over-cavity-displacement.csv plotten, fluoreszenzpower von laserpower abziehen
% Vergleich mit Theorie: zu erwarten ist Lasing falls -1 \le g_1\cdot g_2\le 1 [HeNe-Manual] mit g_i = 1 - L/R_i, L is cavity width in .csv file, R_i sind Krümmungsradien der Resonatorspiegel 


\subsection{Wellenlängenselektion}
    \paragraph{Doppelbrechender Kristall als Selektor}

    \paragraph{Fluoreszenzspektrum von Neon}

    \paragraph{Littrow-Prisma als Selektor}

\subsection{Modenselektion}
    
\end{document}