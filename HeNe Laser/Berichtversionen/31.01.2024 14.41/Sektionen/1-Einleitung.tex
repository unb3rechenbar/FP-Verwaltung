\documentclass[../main.tex]{subfiles}

\begin{document}
    A laser (light amplification by stimulated emission of radiation) is a device for the creation and amplification of  coherent light. For this reason, lasers enable us to use light in a highly controlled manner, which is used in communication (light pulses in fibre cables), medicine, spectroscopy research etc.\\ 

    \noindent Lasers are thus an integral part for experimental and industrial optics. However, lasing is a critical phenomenon, meaning there is not much tolerance for error in a laser setup. Even a small change from a working setup (e.g. not perfectly tidy or slightly wrongly tilted mirrors) may cause complete disruption to the processes at work. The main objective of this experiment is hence to familiarize ourselves with how to correctly and efficiently setup a working laser.\\

    \noindent For this purpose, a Helium Neon laser is used. During the course of the experiment its components are used to built different setups for recording different characteristics of the laser light. The recorded power and wavelength spectrum of laser laser light and the recorded occuring laser modes are related to the theory underlying lasers. This way, an intuition is devoloped how to setup a laser and fine-tune it for a specific laser light. Furthermore, an understanding is gained on how different laser setups result in different characteristics of the laser light.

\end{document}