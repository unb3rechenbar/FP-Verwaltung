\documentclass[../main.tex]{subfiles}

\begin{document}
    A laser (light amplification by stimulated emission of radiation) is a device for the creation and amplification of highly coherent light. For this reason, lasers enable us to use light in a highly controlled manner, which is used in communication (light pulses in fibre cables), medicine, spectroscopy research etc. 

    \noindent Generally speaking, a laser has three important components: a pump, an active medium, and a cavity. The active medium is placed inside the cavity (e.g. between two mirrors) and is then exited with the pump. This causes the atoms of the active medium to undergo absorption processes, which in turn enables spontaneous emission to take place.

    If the rate of absorption is higher than the rate of spontaneous emission, a population inversion can take place: more atoms now occupy exited states than atoms occupy the ground state. Photons that are spontaneously emitted have the opportunity to statt oscillating in the cavity (e.g. being reflected back and forth between two mirrors). During this Osillation, the photons can take part in stimulated emission, which is a coherent process and thus produces the feature cohr
\end{document}