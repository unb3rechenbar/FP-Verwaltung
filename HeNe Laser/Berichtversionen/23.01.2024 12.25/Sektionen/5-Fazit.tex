\documentclass[../main.tex]{subfiles}

\begin{document}

    \subsubsection*{Current dependent output power}
        In this experiment we saw an increasing output power with increasing input current, which appeares to saturate everso slightly at the end. To confirm theoretical results we need to take a bigger frame of reference, as the saturation is not clearly visible in the given range. Furthermore we cannot safely assume a functional relationship of these two variables. Therefore, and because we failed to measure the initial offset $P_fl$, we only get an approximate value $P_0 = (8.2087\pm 30.8496)\;\si{\W}$ for the laser threshold.

        Another result was an almost linear decreasing relationship between the input current and the efficiency of the HeNe Laser. 

    \subsubsection*{Displacement analysis}
        In this experiment we saw a linear relationship between the displacement of the active medium and the output power of the HeNe laser in the first few steps. The rapid stagnation of the output power could be explained by the defocused reflextion beams inside the resonator. 


    \subsubsection*{Optical analysis}
        An examination of resonator stability showed that laser power drops with increasing resonator length $L$. The stability threshold length $L_{th}$ for a pair of one plane and one curved mirror (curvature radius \SI{70}{\cm}) was further determined to be $L_{th} = \SI{70.26(18)}{\cm}$, in accordance with theory.

    \subsubsection*{Wavelength selection and fluorescence spectrum}
        After establishing some dependances of laser power, the characteristics of the laser light itself were probed. Using a birefringent crystal is isolate single wavelengths from the laser light, two laser lines at \SI{633.0(10)}{\nm} and \SI{640}{\nm} could be found. A similar isolation using a littrow prism failed.\\
        
        \noindent The fluorescence spectrum of the Helium Neon gas was also observed for with and without lasing. The difference between both spectra showed mainly peaks and a few valleys. One the one hand, the presence of the peaks allowed for the verification of the coherence of laser light. In more detail, four laser lines at \SI{544.0(10)}{\nm}, \SI{612.0(10)}{\nm}, \SI{640.0(10)}{\nm}, and \SI{633.0(10)}{\nm} were identified.

\end{document}