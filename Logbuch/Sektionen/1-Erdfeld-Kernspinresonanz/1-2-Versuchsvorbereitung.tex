\documentclass{subfiles}

\begin{document}


    \begin{Frage}
        Kernspin, magnetisches Moment, gyromagnetisches Verhältnis.
    \end{Frage}
    \begin{Antwort}
        Unter dem \href{https://de.wikipedia.org/wiki/Kernspin}{\emph{Kernspin}} versteht man den Gesamtdrehimpuls $I$ eines Atomkernes um seinen Massenschwerpunkt $\mu(m)$. Dabei ist $m\in\R^N$ ein Tupel aller Massen der beteilligten Protonen und Neutronen, sowie $I = \sum_{i\in[N]} (S_i + L_i)$ mit dem Spin- und Bahndrehimpuls(operator) $S_i$ und $L_i$ des $i$-ten Teilchens. Seine zugehörige Quantenzahl $\mci$ kann Werte aus $\{n/2:n\in\N\}$ annehmen. Seine Norm ist bestimmt durch $\dabs{I}{} = \hbar\cdot\sqrt{\mci\cdot (\mci+1)}$.\\
        
        Das \href{https://de.wikipedia.org/wiki/Gyromagnetisches_Verhältnis}{\emph{gyromagnetische Verhältnis}} stellt einen Proportionalitätsfaktor zwischen dem magnetischen Moment $\mu$ und dem Drehimpuls $L$ eines Teilchens dar. Es gilt dabei $\mu = \gamma L$ mit $\gamma\in\R$ und Einheit $\si{\ampere\second\per\kg}$ als gyromagnetisches Verhältnis.\\

        Durch das \href{https://de.wikipedia.org/wiki/Magnetisches_Dipolmoment}{\emph{Magnetische Dipolmoment}} wird die Stärke und Richtung eines magnetischen Dipols in der Dimension $\si{\ampere\metre\squared}$ angegeben. 
    \end{Antwort}


    \begin{Frage}
        Kern-Zeeman-Effekt, Energiedifferenz der Zustände, Vergleich mit anderen Anregungszuständen im Atom.
    \end{Frage}
    \begin{Antwort}
        Bei einem Atomkern kann ebenfalls der \emph{anomale Zeeman-Effekt} auftreten, obwohl die Kernmomente um ca. $10^3$ Größenordnungen kleiner als in der Atomhülle ist. Der anomale Zeemaneffekt betrachtet bei anliegendem (homogenen) Magnetfeld die Spin-Bahn-Kopplung und den Teilcheneigenen Spin zu einem Gesamtdrehimpuls $J = L + S$. Daraus resultiert eine Eigenwertaufspaltung des neuen Hamiltonoperators.
    \end{Antwort}
\end{document}