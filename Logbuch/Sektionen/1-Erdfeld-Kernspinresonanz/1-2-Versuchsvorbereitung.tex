\documentclass{subfiles}

\begin{document}


    \begin{Frage}
        Kernspin, magnetisches Moment, gyromagnetisches Verhältnis.
    \end{Frage}
    \begin{Antwort}
        Unter dem \href{https://de.wikipedia.org/wiki/Kernspin}{\emph{Kernspin}} versteht man den Gesamtdrehimpuls $I$ eines Atomkernes um seinen Massenschwerpunkt $\mu(m)$. Dabei ist $m\in\R^N$ ein Tupel aller Massen der beteilligten Protonen und Neutronen, sowie $I = \sum_{i\in[N]} (S_i + L_i)$ mit dem Spin- und Bahndrehimpuls(operator) $S_i$ und $L_i$ des $i$-ten Teilchens. Seine zugehörige Quantenzahl $\mci$ kann Werte aus $\{n/2:n\in\N\}$ annehmen. Seine Norm ist bestimmt durch $\dabs{I}{} = \hbar\cdot\sqrt{\mci\cdot (\mci+1)}$.\\
        
        Das \href{https://de.wikipedia.org/wiki/Gyromagnetisches_Verhältnis}{\emph{gyromagnetische Verhältnis}} stellt einen Proportionalitätsfaktor zwischen dem magnetischen Moment $\mu$ und dem Drehimpuls $L$ eines Teilchens dar. Es gilt dabei $\mu = \gamma L$ mit $\gamma\in\R$ und Einheit $\si{\ampere\second\per\kg}$ als gyromagnetisches Verhältnis.\\

        Durch das \href{https://de.wikipedia.org/wiki/Magnetisches_Dipolmoment}{\emph{Magnetische Dipolmoment}} wird die Stärke und Richtung eines magnetischen Dipols in der Dimension $\si{\ampere\metre\squared}$ angegeben. 
    \end{Antwort}


    \begin{Frage}
        Kern-Zeeman-Effekt, Energiedifferenz der Zustände, Vergleich mit anderen Anregungszuständen im Atom.
    \end{Frage}
    \begin{Antwort}
        Bei einem Atomkern kann ebenfalls der \emph{anomale Zeeman-Effekt} auftreten, obwohl die Kernmomente um ca. $10^3$ Größenordnungen kleiner als in der Atomhülle ist. Der anomale Zeemaneffekt betrachtet bei anliegendem (homogenen) Magnetfeld die Spin-Bahn-Kopplung und den Teilcheneigenen Spin zu einem Gesamtdrehimpuls $J = L + S$. Daraus resultiert eine Eigenwertaufspaltung des neuen Hamiltonoperators.
    \end{Antwort}


    \begin{Frage}
        Drehimpuls, $B_0$, $B_1$, Drehmoment, Präzession, Larmor-Frequenz.
    \end{Frage}
    \begin{Antwort}
        Der \href{https://de.wikipedia.org/wiki/Drehimpuls}{\emph{Drehimpuls}} ist ein Maß für die Drehbewegung eines Körpers, beschrieben durch den Abstandsvektor $r\in\R^3$ vom Drehursprung und dem Impuls $p\in\R^3$ des Körpers. Es gilt $L := r\times p$.\\

        Das \href{https://de.wikipedia.org/wiki/Drehmoment}{\emph{Drehmoment}} ist nach der Konstruktion des Drehimpulses, verwendet jedoch $\dv{t}p(t) =: F(t)$ zu $\mcT := r\times F$. \\

        Die \href{https://de.wikipedia.org/w/index.php?title=Präzession&oldid=235094402}{\emph{Präzession}} ist die Drehbewegung eines rotierenden Körpers um eine Achse, welche selbst rotiert. \\

        Die \href{https://de.wikipedia.org/wiki/Larmorpräzession}{\emph{Larmor-Frequenz}} ist die Frequenz, mit der ein magnetischer Dipol um die Richtung eines äußeren Magnetfeldes präzediert. Sie ist gegeben durch $\omega_L = \gamma B_0$ mit dem gyromagnetischen Verhältnis $\gamma$ und der Stärke des äußeren Magnetfeldes $B_0$.\\
    \end{Antwort}


    \begin{Frage}
        Rotierendes Koordinatensystem (Zerlegung von linear polar. $B_1$ in 2 gegenläufig zirkular polarisierte Komponenten, die um z-Achse rotieren), $B_0$ verschwindet in Resonanz
    \end{Frage}
    \begin{Antwort}
        
    \end{Antwort}


    \begin{Frage}
        $90\si\degree$ und $180\si\degree$ Puls. 
    \end{Frage}
    \begin{Antwort}
        
    \end{Antwort}


    \begin{Frage}
        Bloch Gleichungen und freier Induktionszerfall.
    \end{Frage}
    \begin{Antwort}
        Die \href{}{\emph{Bloch-Gleichungen}} stellen als gewöhnliche Differentialgleichungen erster Ordnung die Änderung der Kern- und Elektronenmagnetisierung $M\in\R^3$ einer \emph{flüssigen} Probe (Gültigkeit für Festkörper nur eingeschränkt) unter Einfluss eines äußeren Magnetfeldes $B(t)\in\R^3$ dar. Als System aufgefasst gilt 
        \[
            \dv{t}M(t) = \gamma\cdot M(t)\times B(t) - \mbbEins_1\cdot\frac{M(t)_1}{T_2} - \mbbEins_2\cdot\frac{M(t)_2}{T_2} - \mbbEins_3\cdot\frac{M(t)_3 - M_0}{T_1}.
        \]
        Dabei stehen $T_1\in\R$ und $T_2\in\R$ für \href{}{\emph{Relaxionszeiten}}, spezifisch beschreibt $T_1$ die Spin-Gitter- und $T_2$ die Spin-Spin-Relaxion. Das DGP besitzt die fastschiefsymmetrische Koeffizientenmatrix $A$ mit affin-linearer Verschiebung $b$ der Form
        \[
            A = \begin{pmatrix}
                -\frac{1}{T_2} & \gamma\cdot B(t)_3 & -\gamma\cdot B(t)_2 \\
                -\gamma\cdot B(t)_3 & -\frac{1}{T_2} & \gamma\cdot B(t)_1 \\
                \gamma\cdot B(t)_2 & -\gamma\cdot B(t)_1 & -\frac{1}{T_1}
            \end{pmatrix},\qquad b = \begin{pmatrix}
                0\\
                0\\
                M_0/T_1
            \end{pmatrix}
        \]
        wobei $\dv{t}M(t) = A\cdot M(t) + b$. \\
    \end{Antwort}

    \begin{Frage}
        Resonanzbedingungen im LCR-Schwingkreis (Empfang des NMR-Signals) [vgl. 1.3.2]
    \end{Frage}
    \begin{Antwort}
        
    \end{Antwort}

    \begin{Frage}
        Fouriertransformation: Beispiele im Zeit- und Frequenzbereich.
        \begin{itemize}[label=$\to$]
            \item Sinus und Cosinus,
            \item Rechteckimpuls und sinc,
            \item exponentielles Abklingen und Lorentz,
            \item Voigt-Profil und Schwebung.
        \end{itemize}
    \end{Frage}
    \begin{Antwort}
        
    \end{Antwort}

    \begin{Frage}
        Fouriertransformation: Aus dem $k$ Raum in den $r$ Raum. 
    \end{Frage}
    \begin{Antwort}
        
    \end{Antwort}

    \begin{Frage}
        Longitudinale Spin-Gitter und transversale Spin-Spin Relaxation. ($T_1$, $T_2$, $T_2^*$) [vgl. 3.3, 4.3]
    \end{Frage}
    \begin{Antwort}
        
    \end{Antwort}

    \begin{Frage}
        Relaxionskontrast: Einfluss paramagnetischer Ionen auf $T_1$ und $T_2$ [vgl. 6.3].
    \end{Frage}
    \begin{Antwort}
        
    \end{Antwort}

    \begin{Frage}
        Hahn- bzw. Spin-Echo, Carr-Purcell- und CPMG-Pulsfolge, $\pi_x$ und $\pi_y$ Winkel, Phase.
    \end{Frage}
    \begin{Antwort}
        
    \end{Antwort}

    \begin{Frage}
        Aufbau der Terranova-Spule.
        \begin{itemize}[label=$\to$]
            \item Kompensation von Magnetfeldinhomogenitäten,
            \item 
        \end{itemize}
    \end{Frage}
    \begin{Antwort}
        
    \end{Antwort}

    \begin{Frage}
        
    \end{Frage}

\end{document}