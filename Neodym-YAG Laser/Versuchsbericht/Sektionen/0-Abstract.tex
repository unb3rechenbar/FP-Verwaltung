\documentclass[../main.tex]{subfiles}

\begin{document}

%\begin{figure}[H]
%    \centering
%    \includegraphics[width=6cm]{Bilddateien/CoffinDance.jpg}
%    \label{fig:myfreshbild}
%\end{figure}

Using a AlGaAs diode laser to stimulate the Nd:YAG crystal we can observe multiple characteristics of thus structure and lasing abilities. By firstly measuring the absorbtion spectrum of the crystal we can determine the parameters for the AlGaAs laser and therefore conclude the optimal frequency of $808\si{\nm}$ for the pumping laser. We also get the opportunity to measure the lifetime of the excited state of the Nd:YAG crystal at $(250\pm 3.6)\si{\micro\s}$ by observing the decay of the population inversion.

Building the laser allows us to observe its output power as a function of the pumping power and to determine the threshold power of the Nd:YAG laser as $\SI{132.7(22)}{\milli\w}$. Furthermore, from this the quantum yield of the laser can be calculated to be $\SI{9.01(77)}{\percent}$, assuming no internal resonator losses. In the case of non-equilibrium, we can observe a spiking in the laser power output, meaning an oscillation over time.

Lastly, using a nonlinear crystal, it is possible to determine the quadratic relationship of the power of an fundamental wave and the power of radiation of doubled frequency. The doubled frequency of the laser light is further within the observable range, meaning we can observe the light modes within the resonator.

\end{document}