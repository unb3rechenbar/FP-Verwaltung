\documentclass[../main.tex]{subfiles}

\begin{document}
    \subsection{Anregungsspektrum des Nd:YAG-Kristalls}
        Um den Nd:YAG-Laser präzise und effizient betreiben zu können, werden zuerst die relevanten Eigenschaften der Laserdiode analysiert.

        \paragraph{Transmissionsspektren des Kristalls unter Diodenstrom.} Die Besetzungsinversion der Laserdiode ist zwischen zwei Zustäned mit einer Bandlücke. Wie für Halbleiter üblich, ist die Breite dieser Bandlücke temperaturabhängig. Damit geht auch eine Abhängigkeit vom Injektionsstrom daher, welcher für

        \paragraph{Kennlinie für Konstante Wellenlänge.}

    \subsection{Lebensdauer des $\,^4F_{3/2}$-Zustands}

    \subsection{Aufbau des Nd:YAG-Kristalls.}

    \subsection{Laserleistung als Funktion der Pumpleistung}

    \subsection{Laserausgangsleistung und Laserschwelle}

    \subsection{Spiking beim Laser-Einschaltvorgang}

    \subsection{Phänomene bei Frequenzverdopplung}
        \paragraph{Lasermoden im Resonator.}
        
        \paragraph{Intensität des frequenzverdoppelten Lichts als Funktion der Pumpleistung.}

\end{document}