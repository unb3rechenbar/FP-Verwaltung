\documentclass[../main.tex]{subfiles}

\begin{document}
    Durch zwei angelegte Stromwerte an den pumpenden Diodenlaser erhielten wir zunächst eine grobe Abschätzung der Temperaturabhängigkeit der gemessenen Ausgangsleistung, mit welcher wir die Antiproportionalität der beiden Größen feststellen konnten. Durch Rasterung über einen breiteren Leistungsbereich erhielten wir die theoretisch erwartete Kennlinie als Verbindung der beiden Größen zur konstanten Erzeugung der benötigten $808\si{\nm}$ Laserlichtwellenlänge. \\

    Die Untersuchung des angeregten Laserniveaus durch pulsierte Stromzufuhr in den Pumplaser ergab einen erwarteten exponentiellen Zerfall der Besetzungsinvasion, durch welchen wir in guter RSS-Näherung auf eine Lebensdauer von $0.25\si{\ms}$ schließen konnten. \\

    Bei der Untersuchung des Transmissionsspektrums des vollständig aufgebauten Nd:YAG Lasers konnten wir einen antiproportionalen Graphenverlauf relativ zum eingangs aufgenommenen Absorptionsspektrum feststellen. Die Minima des Absorptionsspektrums liegen dabei bis auf Skalierung bei den Maxima des Transmissionsspektrums. \\

    Im Falle der Laserausgangsleistung des Nd:YAG Lasers als Funktion der Pumpleistung erhalten wir eine gute erwartete lineare Abhängigkeit, mit welcher wir die Laserschwelle bei $P_{th} =  \si{\mW}$ bestimmen konnten. \\

    Bei dem Vergleich des Nd:YAG Kristalls im Resonatoraufbau mit dem Kristall ohne Resonator liegen die Maxima komplementär zu den Minima an den erwarteten Temperaturwerten. 

    Bei der frequenzverdoppelten Abhängigkeitsuntersuchung der Laserausgangsleistung zur Pumpleistung können wir zwar eine quadratische Abhängigkeit erahnen, jedoch aufgrund mangelnder Messpunkte nicht deutlich bestätigen. Hier ist eine Messwiederholung mit mehr Messpunkten empfohlen. 
\end{document}