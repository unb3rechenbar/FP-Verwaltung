\documentclass[../main.tex]{subfiles}

\begin{document}
    
% AUSWERTUNGSPUNKTE
% zu 2): @Tom (fertig) 1H-Signal des Wasserprobe + Fouriertransformation (FT) plotten; Larmorfrequenz angeben
    \subsection{Initiale Messung und Larmor Frequenzbestimmung}\label{subsec:2:InitialeMessung}
        \subfile{Auswertungsteile/2.tex}

% zu 3): @Tom (fertig) Rauschen + FT plotten; Amplitude vom Rauschsignal bestimmen
    \subsection{Rauschanalyse}\label{subsec:3:Rauschanalyse}
        \subfile{Auswertungsteile/3.tex}

% zu 4): @Tom (fertig) Resonanzfrequenz über Kapazität plotten + mit theoretischem FIT vergleichen
    \subsection{LCR Resonanzanalyse}\label{subsec:4:LCRResonanzanalyse}
        \subfile{Auswertungsteile/4.tex}

% zu 5): @Tom (aktuelle Arbeit) Plots für Optimierung der Parameter: für Shimming, die Kapazität C, Länge des B1-Pulses; optimale Parameter angeben; H1-Signal + FT plotten
    \subsection{Parameteroptimierung}\label{subsec:5:Parameteroptimierung}
        \subfile{Auswertungsteile/5.tex}

% zu 7): @Tiark Diskussion des FID aus 5): insbesondere Amplitude, Linienbreite, Fläche unter Kurve, Signal-zu-Rauschverhältnis, reelles und imaginäres Signal 


\subsection{Charakterisierung des optimierten FID} %evtl. als subsub in Toms vorherigen Kapitel
    \subfile{Auswertungsteile/7.tex}
     
    

    % ALLES GUTE ZUM GEBURTSTAG! 
    %Dankeschön!

% zu 8): Signal bei Polarisationspuls plotten + theoretischer Fit; Relaxionszeit T1 berechnen
    \subsection{Bestimmung der longitudinalen Spin-Gitter Relaxionszeit}\label{subsec:8:BestimmungDerLongitudinalenSpinGitterRelaxionszeit}
        \subfile{Auswertungsteile/8.tex}

% zu 9): Hahn-Echo + FT plotten für optimales Shimming und kleineres/ größeres x-Shimming; Amplitude und Integral der FT diskutieren
\subsection{Hahn-Echo}
    \subfile{Auswertungsteile/9.tex}

% zu 10): @Tom (fertig) T2 Zeiten extrahieren und Beispielplots einbinden 
    \subsection{Phasenwinkelvariation}
        \subfile{Auswertungsteile/10.tex}

% zu 11): @TIark
\subsection{Relaxationskontrast}
    \subfile{Auswertungsteile/11.tex}   
    
\subsection{Magnetresonanztomographie (MRI)}
    \subfile{Auswertungsteile/12.tex}
        
        
\end{document}