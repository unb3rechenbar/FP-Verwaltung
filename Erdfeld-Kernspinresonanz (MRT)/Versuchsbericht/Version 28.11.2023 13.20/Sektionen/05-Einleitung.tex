\documentclass[../main.tex]{subfiles}
\begin{document}

NMR bezeichnet die Manipulierung von Kernspins durch magnetische Pulse, üblicherweise im RF-Bereich\footnote{In diesem Experiment werden jedoch ULF-Pulse genutzt.}. Ein statisches Hintergrundfeld sorgt für eine Ausrichtung der Spins bzw. für eine Magnetisierung der Probe, welche durch die erwähnten Pulse zur Präzession gebracht werden kann. Dies wird in Form eines FID-Pulses gemessen und kann zum Verständnis des Art der vorhanden Kerne, deren räumlichen Verteilung und Interaktion mit der Umgebung der Probe führen. Deshalb findet NMR auch große medizinische Anwendung im Gebiet von MRI.

\end{document}