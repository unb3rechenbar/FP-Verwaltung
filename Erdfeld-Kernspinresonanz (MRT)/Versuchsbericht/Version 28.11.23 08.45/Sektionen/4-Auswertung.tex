\documentclass[../main.tex]{subfiles}

\begin{document}
    
% AUSWERTUNGSPUNKTE
% zu 2): @Tom (fertig) 1H-Signal des Wasserprobe + Fouriertransformation (FT) plotten; Larmorfrequenz angeben

% zu 3): @Tom (fertig) Rauschen + FT plotten; Amplitude vom Rauschsignal bestimmen

% zu 4): @Tom (fertig) Resonanzfrequenz über Kapazität plotten + mit theoretischem FIT vergleichen

% zu 5): @Tom (aktuelle Arbeit) Plots für Optimierung der Parameter: für Shimming, die Kapazität C, Länge des B1-Pulses; optimale Parameter angeben; H1-Signal + FT plotten

% zu 7): @Tiark Diskussion des FID aus 5): insbesondere Amplitude, Linienbreite, Fläche unter Kurve, Signal-zu-Rauschverhältnis, reelles und imaginäres Signal 


\subsection{Charakterisierung des optimierten FID} %evtl. als subsub in Toms vorherigen Kapitel
    \subfile{Auswertungsteile/7.tex}
     
    

    % ALLES GUTE ZUM GEBURTSTAG! 
    %Dankeschön!

% zu 8): Signal bei Polarisationspuls plotten + theoretischer Fit; Relaxionszeit T1 berechnen

% zu 9): Hahn-Echo + FT plotten für optimales Shimming und kleineres/ größeres x-Shimming; Amplitude und Integral der FT diskutieren

% zu 10): @Tom (fertig) T2 Zeiten extrahieren und Beispielplots einbinden 

% zu 11): @TIark
\subsection{Relaxationskontrast}
    \subfile{Auswertungsteile/11.tex}   
    
\subsection{Magnetresonanztomographie (MRI)}
    \subfile{Auswertungsteile/12.tex}
        
        
\end{document}